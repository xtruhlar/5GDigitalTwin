\documentclass[12pt,a4paper]{article}
\usepackage[utf8]{inputenc}
\usepackage[slovak]{babel}
\usepackage{graphicx}
\usepackage[a4paper,margin=2.5cm]{geometry}
\usepackage{setspace}
\renewcommand{\baselinestretch}{1.2} % Adjust line spacing

\begin{document}
\pagestyle{empty}

% Header section
\begin{minipage}{0.6\textwidth}
    \raggedright
    \textbf{Slovenská technická univerzita v Bratislave} \\
    \textbf{Ústav počítačového inžinierstva a aplikovanej informatiky}
\end{minipage}%
\hfill
\begin{minipage}{0.35\textwidth}
    \raggedright
    \textbf{Fakulta informatiky a informačných technológií} \\
    \textbf{2024/2025}
\end{minipage}

\vspace{0.25cm}

\begin{center}
    \includegraphics[width=0.2\textwidth]{assets/images/logo-fiit.png}
\end{center}

\vspace{0.25cm}

\begin{center}
    \Large\textbf{ZADANIE BAKALÁRSKEJ PRÁCE}
\end{center}

\vspace{0.25cm}

% Content section
\noindent
{Autor práce:} Dávid Truhlář \\
{Študijný program:} informatika \\
{Študijný odbor:} informatika \\
{Evidenčné číslo:} FIIT-16768-120897 \\
{ID študenta:} 120897 \\
{Vedúci práce:} Ing. Matej Petrík \\
{Vedúci pracoviska:} Ing. Katarína Jelemenská, PhD. \\
{Názov práce:} \textbf{Výskum v oblasti technológie digitálneho dvojčaťa} \\
{Jazyk, v ktorom sa práca vypracuje:} slovenský jazyk

\vspace{0.5cm}

% Task specification
\noindent
\textbf{Špecifikácia zadania:}

\noindent
Digitálne dvojča je inovatívna technológia, ktorá mení spôsob, akým chápeme a interagujeme s fyzickými objektami, procesmi či systémami. Digitálne dvojča je virtuálna alebo digitálna kópia fyzického objektu, systému alebo procesu, pričom sa snaží zachytiť a kopírovať čo najpresnejšie jeho vlastnosti. Technológia digitálneho dvojčaťa zohráva kľúčovú roľu v Priemysle 4.0 kvôli možnostiam monitorovania, simulácie a automatizácie v reálnom čase. Vďaka týmto možnostiam využitie tejto technológie umožňuje nové úrovne inovácie a optimalizácie naprieč rôznymi odvetviami a taktiež má potenciál akcelerovať vývoj v týchto oblastiach. Preskúmajte využitie technológie digitálneho dvojčaťa a jej aplikácie. Zamerajte sa na preskúmanie konceptu digitálneho dvojčaťa, na spôsob jeho implementácie a taktiež aj na dostupné technológie na jeho tvorbu. Vytvorte prehľad dostupných riešení, navrhnite spôsob vytvorenia jednoduchého digitálneho dvojčaťa, ktoré bude kópiou fyzického objektu alebo systému. Navrhnuté riešenie implementujte a overte jeho funkčnosť. Literatúra: Singh, Maulshree, et al. “Digital twin: Origin to future.“ Applied System Innovation 4.2 (2021): 36. Crespi, Noel, Adam T. Drobot, and Roberto Minerva. \textit{The Digital Twin}. Cham: Springer International Publishing, 2023. Jones, David, et al. “Characterising the Digital Twin: A systematic literature review.“ CIRP journal of manufacturing science and technology 29 (2020): 36-52.\\
\\
{Rozsah práce:} 40 \\
{Termín odovzdania práce:} 12. 05. 2025

\end{document}
