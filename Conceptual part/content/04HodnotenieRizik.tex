\section{Hodnotenie rizík}
Implementácia DT 5G siete v kombinácii s klasifikačným modelom umelej inteligencie prináša niekoľko potenciálnych rizík, ktoré by mohli ovplyvniť presnosť, stabilitu a praktickú využiteľnosť riešenia \cite{ML_traffic}.

Jedným z najvýznamnejších rizík je nedostatočná kvalita vstupných dát. Vzhľadom na to, že systém sa bude spoliehať na syntetické metriky generované v simulovanom prostredí, existuje riziko, že tieto dáta nebudú dostatočne reprezentatívne pre správanie reálnej siete. To by mohlo negatívne ovplyvniť schopnosť modelu generalizovať. Ako mitigácia sa navrhuje testovanie viacerých variantov simulácií, aplikácia metód validácie \cite{Nguyen} (napr. krížová validácia) a neskôr prípadné doplnenie reálnych dátových vstupov \cite{data_generating}.

Ďalším rizikom je obmedzený čas dostupný na zber a spracovanie dát. Experimentálne simulácie si vyžadujú značné množstvo iterácií a manuálnej prípravy scenárov, čo môže limitovať rozsah a kvalitu datasetu pre modelovanie \cite{USAirForce}. Mitigáciou je dôsledné plánovanie simulácií, skriptovanie opakovaných činností a prioritizácia najrelevantnejších scenárov.

V oblasti softvérovej architektúry je potrebné počítať s možnou nekompatibilitou medzi jednotlivými komponentmi ako Open5GS, UERANSIM, Prometheus, Docker a vlastné skripty. Vzhľadom na to, že ide o voľne dostupné riešenia vyvíjané nezávisle, ich integrácia môže byť nestabilná alebo nedostatočne zdokumentovaná \cite{challenges_human_factor}. Odporúčanou stratégiou je modulárne nasadzovanie systémov, priebežné testovanie a dôsledná kontrola konfigurácií v menších izolovaných častiach.

Ďalším rizikom je nesynchronizácia časových pečiatok medzi komponentmi, čo môže ovplyvniť presnosť dátových okien pre LSTM model. Tento problém je možné mitigovať precíznym logovaním časových pečiatok a synchronizáciou systémového času medzi kontajnermi.

Z pohľadu bezpečnosti a etiky sa pri použití simulovanej siete s fiktívnymi údajmi a v uzavretom prostredí eliminuje riziko úniku osobných údajov či konfiguráčných súborov. Napriek tomu je vhodné minimalizovať citlivosť konfigurácií pomocou .env súborov \cite{Dt_Iot_data_worry_about} a testovať výhradne na syntetických identitách a medzinárodných identitách mobilného účastníka (IMSI).

Všetky uvedené riziká boli zohľadnené pri návrhu architektúry a experimentálnych scenárov tak, aby bola zabezpečená čo najvyššia spoľahlivosť riešenia.