\chapter{Úvod}
%Študenti by mali načrtnúť problematiku, zdôrazniť súčasné výzvy a hlavné zásady a diskutovať o interdisciplinárnych aspektoch. ½ strany
Technológia digitálneho dvojčaťa (Digital Twin, DT) predstavuje jeden z najperspektívnejších prístupov v oblasti moderného softvérového inžinierstva \cite{DimensionOfDTAplication} a telekomunikácií \cite{AplicationsOfDT}. DT umožňuje vytvárať virtuálne repliky fyzických systémov, ktoré v reálnom čase zrkadlia ich správanie pomocou obojsmernej výmeny dát \cite{real_time}. Takéto systémy nachádzajú uplatnenie v rôznych doménach vrátane priemyselnej výroby \cite{manufacturing}, zdravotníctva \cite{siemens_helthcare}, dopravy a energetiky, kde umožňujú monitorovanie v reálnom čase, prediktívne riadenie a optimalizáciu procesov.

V oblasti mobilných sietí, najmä 5G, poskytujú DT nové možnosti v simulácii sieťového správania, včasnej detekcii anomálií a optimalizácii alokácie zdrojov \cite{5gandbeyond}. 5G infraštruktúry sa vyznačujú vysokou mierou heterogenity, dynamiky a nárokmi na kvalitu služieb (QoS), čo výrazne komplikuje ich spravovanie a testovanie \cite{AplicationsOfDT}. DT v tomto kontexte umožňuje testovanie rôznych scenárov v bezpečnom virtuálnom prostredí bez rizika výpadku služby či narušenia integrity dát.

Kľúčovou výzvou pri návrhu DT pre 5G siete je však zabezpečenie dostatočnej vernosti simulácie. Otázna zostáva najmä generalizovateľnosť modelov trénovaných výhradne na syntetických dátach voči reálnym podmienkam, ktoré sú komplexnejšie, menej predvídateľné a menej kontrolované. Cieľom tejto bakalárskej práce je preskúmať túto výzvu prostredníctvom návrhu a implementácie jednoduchého digitálneho dvojčaťa pre 5G sieť, postaveného na open-source nástrojoch Open5GS \cite{open5gs} a UERANSIM \cite{ueransim}, a vyhodnotiť schopnosť modelov klasifikovať správanie siete v reálnom čase na základe synteticky získaných metrík.