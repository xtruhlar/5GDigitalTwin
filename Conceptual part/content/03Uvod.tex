\thispagestyle{empty}
\chapter{Úvod}

Technológia digitálneho dvojčaťa (DT) predstavuje perspektívny prístup v modernom softvérovom inžinierstve \cite{DimensionOfDTAplication} a telekomunikácií \cite{AplicationsOfDT}. DT umožňuje vytvárať virtuálne kópie fyzických systémov, ktoré v reálnom čase zrkadlia ich správanie pomocou obojsmernej výmeny dát \cite{real_time}. DT nachádzajú praktické uplatnenie v rôznych doménach vrátane výroby \cite{manufacturing}, zdravotníctva \cite{siemens_helthcare} a energetiky, kde umožňujú monitorovanie v reálnom čase, prediktívne riadenie a optimalizáciu procesov \cite{AplicationsOfDT}.

V oblasti sietí, najmä sietí piatej generácie (5G), poskytujú DT nové možnosti v simulácii sieťového správania, včasnej detekcii anomálií a optimalizácii alokácie zdrojov \cite{5gandbeyond}. 5G infraštruktúry sa vyznačujú vysokou heterogenitou, dynamikou a nárokmi na kvalitu služieb (QoS), čo komplikuje ich správu a testovanie \cite{AplicationsOfDT}. DT v tomto kontexte umožňuje testovanie rôznych scenárov vo virtuálnom prostredí bez rizika výpadku služby či narušenia integrity dát. Kľúčovou výzvou pri návrhu DT pre 5G siete je však zabezpečenie dostatočnej vernosti simulácie. Otázna zostáva najmä generalizovateľnosť modelov trénovaných iba na syntetických dátach voči reálnym podmienkam, ktoré sú komplexnejšie a menej predvídateľné. Cieľom tejto práce je preskúmať túto oblasť prostredníctvom návrhu a implementácie jednoduchého DT pre 5G sieť, postaveného na voľne dostupných nástrojoch Open5GS \cite{open5gs} a UERANSIM \cite{ueransim}, a vyhodnotiť schopnosť modelov klasifikovať správanie siete v reálnom čase na základe synteticky generovaných metrík.
