\setcounter{figure}{0}
\setcounter{listing}{0}

\chapter{Používateľská príručka}
\label{appendix:userguide}
\pagenumbering{arabic}
\renewcommand*{\thepage}{B-\arabic{page}}

\begin{refsegment}

\section*{Používateľská príručka}

\subsection*{Platformy a kompatibilita}
\begin{itemize}
  \item macOS Sequoia 15.4
  \item Ubuntu 22.04.3 LTS
  \item Windows 11 Home s nainštalovaným WSL2
\end{itemize}

\subsection*{Požiadavky}
Git (verzia 2.39.5), Docker (verzia 28.0.1)


\subsection*{Inštalácia a spustenie systému}
\begin{enumerate}
  \item Klonujte repozitár a prejdite do adresára projektu:
  \begin{verbatim}
  git clone https://github.com/xtruhlar/5GDigitalTwin.git
  cd 5GDigitalTwin/Implementation
  \end{verbatim}

  \item Vytvorte Docker obrazy:
  \begin{verbatim}
  cd ./base
  docker build -t docker_open5gs .

  cd ../ueransim
  docker build -t docker_ueransim .

  cd ..
  \end{verbatim}

  \item Nastavte premenné prostredia:
  \begin{verbatim}
  cp .env.example .env

  set -a
  source .env
  set +a
  \end{verbatim}

  \item Spustite celý systém pomocou Docker Compose:
  \begin{verbatim}
  docker compose -f deploy-all.yaml up --build -d
  \end{verbatim}

  \item Naimportujte preddefinovaných účastníkov do MongoDB:
  \begin{verbatim}
  docker exec -it mongo mkdir -p /data/backup
  docker cp ./mongodb_backup/open5gs mongo:/data/backup/open5gs
  docker exec -it mongo mongorestore \ 
    --uri="mongodb://localhost:27017" \
    --db open5gs /data/backup/open5gs
  \end{verbatim}

  \item Overte funkčnosť jadra siete cez Open5GS WebUI: \\ 
  Otvorte \url{http://localhost:9999}  
  Prihlásenie: \\ Meno: \texttt{admin} \\ Heslo: \texttt{1423}

  \item Spustite UERANSIM gNB:
  \begin{verbatim}
  docker compose -f nr-gnb.yaml -p gnodeb up -d && \
    docker container attach nr_gnb
  \end{verbatim}

  \item Pripojte registrované zariadenie (UE):
  \begin{verbatim}
  docker compose -f nr-UEs/nr-ue1.yaml -p ues up --build -d
  \end{verbatim}

  \item Vizualizácia klasifikovaného stavu 5G siete:  
  Otvorte Grafanu na \url{http://localhost:3000}  
  Prihlásenie:\\ Meno: \texttt{open5gs} \\ Heslo: \texttt{open5gs}  
  \\Potom kliknite na \texttt{Dashboards → Current state Dash}
\end{enumerate}

% Bibliography
\printbibliography[heading=referencessec,segment=\therefsegment,resetnumbers=true]

\end{refsegment}