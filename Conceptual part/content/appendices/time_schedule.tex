% Time schedule
\thispagestyle{empty}

\ifx\FIITlagEN\undefined
\chapter{Harmonogram práce}
\else
\chapter{Project task schedule}
\fi

\pagenumbering{arabic}
\renewcommand*{\thepage}{A-\arabic{page}}

\section{Zimný semester}

\begin{tabular}{|p{2.7cm}||p{10.4cm}|}
\hline
1. - 5. týždeň    & Konzultácie, rešerš problematiky  \\
\hline
6. týždeň    & Formulácia problému \\
\hline
7. týždeň   & Technický literárny prehľad \\
\hline
8. týždeň                       & Konzultovanie, Zamestnateľnosť   \\
\hline
9. - 10. týždeň   & Implementovanie pripomenutých zmien,  Hodnotenie rizík, Udržateľnosť a environmentálny dopad  \\
\hline
11. týždeň  & Návrh riešenia na vysokej úrovni \\
\hline
12. týždeň & Úvod, Konzultovanie, Odovzdávanie BP 1 \\
\hline
\end{tabular}

\subsection{Vyjadrenie k harmonogramu}
Harmonogram sa dodržal, čo prispelo k systematickému postupu pri spracovaní BP. Pravidelné konzultácie s vedúcim práce zohrali kľúčovú úlohu pri jej realizácii. Vedúci poskytoval pripomienky a odporúčania, na základe ktorých sa jednotlivé časti práce mohli upraviť do súčasnej podoby. Tento proces mi umožnil efektívne riešiť prípadné nedostatky a zabezpečiť súčasnú kvalitu výsledného dokumentu.


\section{Letný semester}

\begin{tabular}{|p{2.7cm}||p{6.2cm}||p{4.2cm}|}
\hline
Týždeň & Bakalárska práca & Článok \\
\hline
1. - 2. týždeň & Tímová práca, diverzita a inklúzia & Súvisiaca práca  \\
\hline
3. - 4. týždeň & Open5GS, UERANSIM, Zber dát & Úvod, Metodológia, Kľúčové slová \\
\hline
5. - 7.  týždeň  & Zber dát, Štúdium LSTM & Opis Modelu  \\
\hline
8. - 10. týždeň & ML model, trénovanie, Reprodukovateľnosť a integrácia & Evaluácia a Diskusia   \\
\hline
11. týždeň & Technický abstrakt, laický abstrakt & Záver, Abstrakt  \\
\hline
12. týždeň &  Zhrnutie, úpravy, konzultovanie & Budúci výskum \\
\hline
\end{tabular}

\subsection{Vyjadrenie k harmonogramu}
Harmonogram sa vo všeobecnosti podarilo dodržať, hoci práca na niektorých úlohách bola náročnejšia, než sa pôvodne očakávalo. Najmä zber dát a implementácia riešenia si vyžadovali viac času a experimentovania. Vďaka systematickému postupu, pravidelnej práci a priebežným konzultáciám sa však podarilo udržať plánovaný smer a postupne napredovať. Tento prístup umožnil priebežné identifikovanie problémov, ich efektívne riešenie a priebežné zlepšovanie kvality výstupov, čo sa pozitívne prejavilo aj na výslednom spracovaní bakalárskej práce a sprievodného článku.