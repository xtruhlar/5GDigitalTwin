\chapter{Laický abstrakt}
Digitálne dvojča predstavuje virtuálny model reálneho systému, ktorý umožňuje sledovať a analyzovať jeho správanie v reálnom čase. V posledných rokoch si táto technológia našla uplatnenie v priemysle, doprave či zdravotníctve. V tejto práci sa venujeme vytvoreniu digitálneho dvojčaťa pre 5G sieť s cieľom porozumieť, ako sa správa pri rôznych používateľských scenároch.

Pomocou voľne dostupných nástrojov sme vytvorili softvérové prostredie, ktoré simuluje správanie reálnej mobilnej siete. Následne sme zbierali dáta nielen zo simulácií, ale aj z reálnych zariadení pripojených do 5G siete. Tieto údaje zahŕňali napríklad počet pripojených používateľov či objem prenesených dát.

Hlavným cieľom bolo overiť, či je možné natrénovať model umelej inteligencie na simulovaných dátach a použiť ho na použiť ho na rozpoznávanie typov sieťovej prevádzky v reálnej sieti. Takýto prístup je bezpečný, flexibilný a umožňuje testovanie aj zriedkavých alebo extrémnych scenárov bez zásahu do reálnej prevádzky.

Aj keď vytvorený systém dokáže spoľahlivo zbierať dáta, priebežne ich analyzovať a v reálnom čase aktualizovať svoj model, jeho schopnosť správne rozpoznať správanie v reálnej sieti je zatiaľ obmedzená. Táto práca tak približuje nielen výhody digitálneho dvojčaťa, ale aj výzvy, ktoré je potrebné riešiť pri nasadení takýchto systémov v reálnych 5G sieťach.