\chapter{Laický abstrakt}

\par{
%Laický abstrakt je stručné vysvetlenie výskumnej práce napísané jednoduchým, netechnickým jazykom. Účelom je preukázať plynulosť študenta v komunikačných zručnostiach voči širokej verejnosti zrozumiteľne. Laický abstrakt by mal mať maximálne 250 slov a je neštruktúrovaný. \btHL1/2 strany

Digitálne dvojča predstavuje virtuálny model reálneho systému, ktorý umožňuje sledovať a analyzovať jeho správanie v reálnom čase. V posledných rokoch si táto technológia našla uplatnenie v priemysle, doprave či zdravotníctve. V tejto práci sa venujem vytvoreniu digitálneho dvojčaťa pre 5G sieť s cieľom porozumieť, ako sa správa pri rôznych používateľských scenároch.

Pomocou voľne dostupných nástrojov som vytvoril softvérové prostredie, ktoré simuluje správanie reálnej mobilnej siete. Následne som zbieral dáta nielen zo simulácií, ale aj z reálnych zariadení pripojených do 5G siete. Tieto údaje zahŕňali napríklad počet pripojených používateľov či objem prenesených dát.

Hlavným cieľom bolo overiť, či je možné natrénovať model umelej inteligencie na simulovaných dátach a použiť ho na rozpoznávanie situácií v reálnej sieti. Takýto prístup je bezpečný, flexibilný a umožňuje testovanie aj zriedkavých alebo extrémnych scenárov bez zásahu do reálnej prevádzky.

Výsledný systém dokáže klasifikovať správanie siete podľa typických vzorcov a vytvára priestor pre budúce nasadenie monitorovania a správy sietí. 
}

