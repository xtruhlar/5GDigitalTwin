\section{Udržateľnosť a environmentálny dopad}
DT predstavujú účinný nástroj na optimalizáciu prevádzkových procesov – v porovnaní s fyzickým testovaním umožňujú výrazne znížiť spotrebu zdrojov, produkciu odpadu aj operačné riziká \cite{enviro_raw_materials}. V kontexte návrhu DT 5G siete, implementovaného v rámci tejto práce, bola environmentálna udržateľnosť reflektovaná v niekoľkých rovinách.

Z pohľadu výpočtovej náročnosti bola celá infraštruktúra navrhnutá s cieľom minimalizovať energetickú záťaž. Celý systém bol prevádzkovaný na lokálnom výpočtovom zariadení bez využitia centralizovaných cloudových služieb či grafických akcelerátorov (GPU). Kombinácia kontajnerizovaných služieb (Docker Compose) a voľne dostupných softvérových nástrojov (Open5GS, UERANSIM, Prometheus, Grafana) umožnila zabezpečiť nízku spotrebu zdrojov \cite{docker_enviro} pri zachovaní dostatočnej funkcionality pre experimentálne overenie konceptu \cite{docker_enviro_2}.

V širšom zmysle sú DT považované za technológiu s potenciálom znižovať uhlíkovú stopu prostredníctvom presnejšej správy sietí, prediktívnej údržby a minimalizácie potreby fyzických zásahov do infraštruktúry \cite{DT_edge_networks_IoT, enviro}. Aj keď prediktívne modelovanie nebolo predmetom tejto práce, implementovaný systém poskytuje základ, na ktorom je možné budovať systémy podporujúce rozhodovanie s pozitívnym environmentálnym dopadom \cite{malaysia_enviro}.

Navyše, architektúra riešenia je modularizovaná, čo umožňuje selektívnu údržbu a výmenu komponentov bez nutnosti reinštalácie celého systému. Tento návrhový prístup je zároveň v súlade s princípmi zeleného softvérového inžinierstva, ktoré zdôrazňujú energetickú efektivitu a podporujú dlhodobú udržateľnosť softvérových riešení \cite{modular_sw}.

Navrhnutý systém teda nielen zohľadňuje technologickú efektivitu a experimentálnu hodnotu, ale zároveň reflektuje environmentálne požiadavky pri návrhu a testovaní moderných komunikačných systémov.