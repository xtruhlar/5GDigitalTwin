\section{Tímová práca, diverzita a inklúzia}

Hoci bola bakalárska práca vypracovaná individuálne, viaceré jej časti by nebolo možné realizovať bez odbornej spolupráce s ďalšími osobami a komunitami.

Počas celého vývoja zohrával významnú úlohu školiteľ Ing. Matej Petrík, ktorý poskytoval metodické usmernenia v oblasti DT, návrhu experimentov, spracovania dát a validácie výsledkov. V technických aspektoch súvisiacich s 5G sieťami, konfiguráciou Open5GS a zberom metrických údajov zo siete prebiehala konzultácia s Ing. Matejom Janebom, ktorý sa aktívne podieľal na verifikácii experimentálnej infraštruktúry a asistoval pri získavaní reálneho datasetu v laboratórnych podmienkach.

Dôležitým prvkom riešenia bola aj interakcia s komunitou vývojárov voľne dostupného nástroja UERANSIM. V rámci otvorenej komunikácie bol kontaktovaný hlavný autor projektu Ali Güngör so žiadosťou o objasnenie spracovania RRC Release signálov, ktoré absentovali v oficiálnom repozitári. Spolupráca napomohla lepšiemu pochopeniu obmedzení dostupných nástrojov a zdôraznila význam komunitne vyvíjaného softvéru pri budovaní výskumných riešení.

Riešenie zároveň reflektuje princípy inklúzie tým, že všetky vytvorené skripty a výstupy sú otvorene dokumentované, konfigurovateľné a dostupné bez obmedzenia technologickej platformy či licencie. Vďaka modulárnej architektúre, využitiu voľne dostupných technológií a kontajnerizovanému nasadeniu je výstup práce plne replikovateľný a potenciálne využiteľný aj v ďalších výskumných kontextoch.