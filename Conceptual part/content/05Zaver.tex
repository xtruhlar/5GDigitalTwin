\chapter{Záver}

V rámci tejto bakalárskej práce bol navrhnutý a implementovaný základný model DT pre 5G sieť, ktorý umožňuje simuláciu správania siete a zber dát z fyzickej i simulovanej infraštruktúry. Použitím nástrojov Open5GS a UERANSIM bol vytvorený kontrolovaný experimentálny priestor, v ktorom bolo možné testovať rôzne UC bez zásahu do reálnej siete. 

Výsledky ukazujú, že klasifikačné modely trénované výhradne na syntetických dátach dosahovali výrazne horšiu presnosť pri aplikácii na reálne dáta (44\%), v porovnaní so 96.3\% presnosťou na dátach zo simulácie (pozri Tab. \ref{table:performance}). Tento pokles jasne signalizuje rozdiel medzi reálnym a simulovaným správaním a zdôrazňuje potrebu spoľahlivých metód prenosu znalostí medzi týmito dvoma doménami.

\begin{table}[H]
\centering
\caption{Výkon klasifikácie tried pre syntetické a reálne dáta.}
\label{table:performance}
\begin{tabular}{|lcccc|}
\hline
\multicolumn{1}{|c|}{\textbf{Model}} & \multicolumn{1}{c|}{\textbf{Presnosť}} & \multicolumn{1}{c|}{\textbf{Presnosť tried}} & \multicolumn{1}{c|}{\textbf{F1-Skóre}} & \textbf{Úplnosť} \\ \hline
\multicolumn{5}{|c|}{\textbf{Syntetický dataset}} \\ \hline
\multicolumn{1}{|l|}{\textit{Základný Model}} & \multicolumn{1}{c|}{0.949} & \multicolumn{1}{c|}{0.950} & \multicolumn{1}{c|}{0.949} & 0.949 \\ \hline
\multicolumn{1}{|l|}{\textit{Robustný Model}} & \multicolumn{1}{c|}{0.900} & \multicolumn{1}{c|}{0.905} & \multicolumn{1}{c|}{0.900} & 0.900  \\ \hline
\multicolumn{1}{|l|}{\textit{Model s Normalizáciou Dávky}} & \multicolumn{1}{c|}{0.963} & \multicolumn{1}{c|}{0.964} & \multicolumn{1}{c|}{0.963} & 0.963 \\ \hline
\multicolumn{1}{|l|}{\textit{Model s Pozornosťou}} & \multicolumn{1}{c|}{0.914} & \multicolumn{1}{c|}{0.918} & \multicolumn{1}{c|}{0.914} & 0.914 \\ \hline
\multicolumn{5}{|c|}{\textbf{Reálny dataset}} \\ \hline
\multicolumn{1}{|l|}{\textit{Základný Model}} & \multicolumn{1}{c|}{0.14} & \multicolumn{1}{c|}{0.06} & \multicolumn{1}{c|}{0.04} & 0.14 \\ \hline
\multicolumn{1}{|l|}{\textit{Robustný Model}} & \multicolumn{1}{c|}{0.44} & \multicolumn{1}{c|}{0.25} & \multicolumn{1}{c|}{0.29} & 0.44 \\ \hline
\multicolumn{1}{|l|}{\textit{Model s Normalizáciou Dávky}} & \multicolumn{1}{c|}{0.21} & \multicolumn{1}{c|}{0.16} & \multicolumn{1}{c|}{0.17} & 0.21 \\ \hline
\multicolumn{1}{|l|}{\textit{Model s Pozornosťou}} & \multicolumn{1}{c|}{0.16} & \multicolumn{1}{c|}{0.03} & \multicolumn{1}{c|}{0.05} & 0.16 \\ \hline
\end{tabular}
\end{table}

Tieto zistenia poukazujú na viaceré limity implementovaného prístupu, ktoré je potrebné otvorene pomenovať. Aj napriek systematickej exploratívnej analýze a dôslednému výberu metrík so stabilným významom naprieč oboma doménami sa ukázalo, že syntetické metriky generované pomocou Open5GS majú obmedzenú schopnosť zachytiť komplexitu reálneho sieťového správania. Chýba im detailná reprezentácia QoS mechanizmov, realistický prenos na aplikačnej vrstve, ako aj podpora mobility, čo výrazne znižuje použiteľnosť týchto simulácií ako základ pre trénovanie modelov s dobrou generalizačnou schopnosťou. Hoci boli testované viaceré architektúry klasifikačných modelov (vrátane mechanizmu s pozornosťou a normalizácie dávky), výkonnosť na reálnych dátach ostávala nízka, čo naznačuje, že problém spočíva predovšetkým v obmedzenej kvalite dát. 

Navyše, použité metriky sa týkali výlučne jadrovej vrstvy, pričom chýbali detailnejšie údaje z RAN vrstvy, ako napríklad parametre rádiového signálu alebo využitie fyzických kanálov. Z pohľadu reálneho experimentálneho zberu bola práca obmedzená na malý počet UE, krátke trvanie meraní, čo mohlo ovplyvniť spoľahlivosť výstupov. Samotné UC boli však navrhnuté tak, aby pokrývali reprezentatívne spektrum bežnej aj hraničnej prevádzky siete: od opakovaných krátkych spojení (napr. periodické získavanie predpovede počasia), cez stabilné dátové toky (streamovanie), až po málo aktívne, no perzistentné relácie (držanie spojenia) či autentizačné chyby. Každý UC tak reprezentuje špecifickú sieťovú záťaž a jeho úspešná klasifikácia poskytuje informáciu o charaktere prevádzky v danom časovom okne.

Práca ukazuje, že klasifikácia v reálnom čase a ladenie modelov je technicky realizovateľné aj v rámci jednoduchého DT, hoci zatiaľ len pri použití syntetických dát. Výsledky tejto práce zároveň naznačujú, že DT môžu nájsť uplatnenie v oblasti riadenia a optimalizácie 5G sietí – predovšetkým ako nástroj na bezpečné testovanie sieťových zmien, včasné odhaľovanie anomálií a trénovanie systémov umelej inteligencie mimo produkčného prostredia. Do budúcnosti by mal byť výskum orientovaný na zvýšenie vernosti simulácií a zlepšenie kvality dát použitých na trénovanie modelov. Za účelom premostenia medzi syntetickými a reálnymi dátami sa ako perspektívna javí kombinácia metrických údajov z Open5GS s dátami zachytenými v podobe paketov (PCAP), čo umožní extrahovať dotatočné charakteristiky sieťového správania a vrstviť informácie naprieč L2–L7. Zároveň je potrebné preskúmať možnosti aplikácie pokročilých doménovo adaptačných techník, ktoré by mohli znížiť závislosť na rozsiahlych reálnych datasetoch.

Výzvou do budúcnosti ostáva rozšíriť prístup v reálnom čase aj na reálne sieťové prostredie a uzavrieť spätnú regulačnú slučku – teda umožniť systému nielen detegovať anomálie v reálnom čase, ale aj autonómne reagovať vhodnou rekonfiguráciou siete alebo upozornením správcu. Takýto obojsmerný tok dát medzi fyzickým a digitálnym prostredím predstavuje ďalší krok smerom k autonómnym, adaptívnym 5G sieťam.