\section{Formulácia problému}
V oblasti 5G sietí predstavujú DT perspektívny nástroj na bezpečné testovanie, monitorovanie a optimalizáciu správania siete. Vysoké nároky 5G na nízku odozvu, spoľahlivosť a flexibilitu riadenia zároveň vyžadujú schopnosť rýchlo identifikovať typické aj neštandardné správanie vrátane anomálií a útokov. Aby bolo možné efektívne trénovať klasifikačné modely bez rizika pre produkčnú infraštruktúru, je nevyhnutné simulovať rôzne scenáre v kontrolovanom prostredí digitálneho dvojčaťa.

Kľúčovým problémom však zostáva otázka, do akej miery môžu modely trénované výhradne na syntetických dátach zo simulovaných prostredí, ako Open5GS a UERANSIM, generalizovať na reálne siete, keďže výrazný nesúlad medzi syntetickými a reálnymi dátami môže obmedziť praktickú využiteľnosť takto vytvorených modelov.

Cieľom tejto bakalárskej práce je preto navrhnúť a implementovať jednoduché DT 5G siete a preskúmať, do akej miery možno synteticky generované metriky využiť na trénovanie modelov schopných klasifikovať správanie v reálnej sieti.