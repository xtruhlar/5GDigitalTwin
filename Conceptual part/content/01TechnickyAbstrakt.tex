%Technický abstrakt je stručný súhrn práce, obvykle obsahujúci okolo 250 slov. Môže byť štruktúrovaný (napr. obsahovať časti ako účel, metódy, výsledky a záver) alebo neštruktúrovaný. Cieľom je, aby sa študent naučil vytvárať stručné technické zhrnutie svojej práce. {\btHL ½ strany}

\chapter{Technický abstrakt}

5G siete predstavujú základ infraštruktúry pre kritické aplikácie s vysokými nárokmi na nízku latenciu a vysokú spoľahlivosť. V tomto kontexte môžu zohrávaať digitálne dvojčatá (Digital Twin) kľúčovú úlohu pri simulácii, monitorovaní a optimalizácii siete v reálnom čase. Cieľom tejto práce je navrhnúť a implementovať klasifikačný systém využívajúci digitálne dvojča 5G siete, ktorý je schopný v reálnom čase identifikovať typ sieťovej prevádzky (Use Case - UC) pomocou metód strojového učenia.

Navrhnutý systém využíva nástroje Open5GS a UERANSIM na generovanie syntetických metrických dát zo šiestich typov sieťovej záťaže, pričom tieto scenáre pokrývajú ako bežné, tak aj anomálne správanie používateľov. Súčasne prebieha zber dát z reálnej siete a porovnanie rozložení metrík pomocou metód ako KL-divergencia a PCA vizualizácie. Po analýze prieniku medzi syntetickým a reálnym priestorom bol navrhnutý robustný výber metrík pomocou kombinácie Random Forest, RFE, SFS a permutačnej importance.

Výsledný model využíva LSTM architektúru optimalizovanú pomocou Batch normalizácie pre klasifikovanie stavu siete v reálnom čaase. Model trénovaný výhradne na syntetických dátach dosiahol na reálnych dátach presnosť ~13\%, ktorá bola následne zlepšená pomocou online fine-tuningu založeného na správne označených UC hodnotách až na ~50\%. Model je nasadený v Docker kontejneroch, načítava real-time metriky a predikcie sú exportované do Prometheus ako vlastné metriky. Celý systém je vizualizovaný v Grafane, vrátane dôveryhodnosti predikcie a verziovania modelu. 

Práca potvrdzuje, že pri dôslednej doménovej analýze je možné trénovať použiteľný model výhradne na syntetických dátach a následne ho adaptovať na produkčné nasadenie v reálnej 5G infraštruktúre.
