\chapter{Technický abstrakt}

Siete piatej generácie (5G) predstavujú základnú infraštruktúru pre aplikácie s prísnymi požiadavkami na odozvu (latency) a spoľahlivosť. Technológia digitálneho dvojčaťa  (DT) má v tomto kontexte potenciál slúžiť ako adaptívna vrstva pre simuláciu, monitorovanie a klasifikáciu sieťovej prevádzky. Táto práca sa zameriava na návrh a implementáciu DT 5G siete, ktoré v reálnom čase analyzuje metriky jadra siete a klasifikuje aktuálny typ prevádzky pomocou rekurentných neurónových sietí (LSTM).

Navrhnuté DT pozostáva z kontajnerizovaného simulačného prostredia založeného na Open5GS a UERANSIM, doplneného o klasifikačný model. V rámci experimentov boli vygenerované syntetické dáta v šiestich definovaných používateľských scenároch (UC) a získané reálne dáta z fyzickej siete. Bol navrhnutý robustný výber metrík (kombináciou metód „Random Forest“, rekurzívne odstraňovaných príznakov a permutačnej dôležitosti) s cieľom identifikovať znaky vhodné pre klasifikáciu v oboch doménach. Modely boli trénované výhradne na syntetických dátach, pričom ich výkon na reálnych dátach bol následne vyhodnotený s jemným doladením aj bez neho.

Napriek vysokej presnosti modelov na syntetických dátach (96,3\%) sa ukázalo, že ich schopnosť generalizácie na reálnu prevádzku je výrazne obmedzená. Bez akéhokoľvek doladenia dosahovali modely na reálnych dátach presnosť len 14-44\%, pričom ani dodatočné ladenie nedokázalo zvýšiť presnosť nad 50\%. To zodpovedá úrovni náhodného tipovania pri šiestich triedach a naznačuje vážny problém doménového prenosu. DT však preukázalo technickú funkčnosť, je schopné v reálnom čase zbierať dáta, spúšťať klasifikáciu a aktualizovať model pomocou jemného doladenia bez prerušenia prevádzky. Výsledky ukazujú, že samotná infraštruktúra DT je technologicky udržateľná, avšak účinné správanie klasifikačného modelu si vyžaduje pokročilejšie techniky adaptácie medzi doménami.
