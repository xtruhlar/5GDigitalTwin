subsection{EDA, predspracovanie a výber príznakov}

Pred samotným trénovaním modelov prebehla exploratívna analýza dát (EDA) a niekoľko krokov predspracovania, ktoré boli identické pre reálne aj syntetické dáta. Analýza sa zameriavala na detekciu chýbajúcich dát, redundantných atribútov a nevhodných korelácií. Anomálne alebo irelevantné metriky (napr. tie s nulovou variabilitou) boli vyradené už v tejto fáze. Vizualizácia korelačnej matice potvrdila existenciu silne redundantných metrík, ktoré by mohli znižovať robustnosť modelu.

Pre účely klasifikácie bola navrhnutá štandardizovaná predspracovateľská pipeline pozostávajúca z:

\begin{itemize}
  \item Doplnenie chýbajúcich hodnôt: pomocou najčastejšej hodnoty (\texttt{mode}) pre každý atribút,
  \item Mapovanie kategórií na číselné hodnoty: pre atribúty \texttt{application}, \texttt{log\_type} a \texttt{current\_uc}, ktoré boli premapované pomocou konfigurovaných slovníkov,
  \item Výber numerických atribútov: zo vstupného dataframe boli vybrané iba číselné stĺpce (okrem \texttt{timestamp} a \texttt{current\_uc}),
  \item Škálovanie atribútov: pomocou \texttt{StandardScaler}, čím sa zabezpečilo nulové stredné hodnoty a jednotková smerodajná odchýlka.
\end{itemize}

Výber príznakov bol realizovaný kombináciou dvoch nezávislých metód — rozhodovacích stromov (\texttt{RandomForestClassifier}) a permutačnej dôležitosti. Cieľom bolo identifikovať príznaky, ktoré sú stabilne informatívne naprieč reálnymi aj syntetickými dátami. Do finálneho modelu boli zaradené len tie príznaky, ktoré spĺňali nasledovné prahové podmienky:

\begin{itemize}
  \item Random Forest importance (syntetické dáta): $\geq 0.01$,
  \item Random Forest importance (reálne dáta): $\geq 0.03$,
  \item Permutačná importance (reálne dáta): $\geq 0.001$.
\end{itemize}

Výsledky dôležitosti vybraných príznakov sú uvedené v Tabuľke \ref{table:feature-selection-synthetic} a \ref{table:feature-selection-real}. Výsledná množina príznakov bola doplnená o dve doménovo významné kategórie: \texttt{application} a \texttt{log\_type}, ktoré boli mapované na číselné hodnoty a zaradené ako doplnkové črty. Výstupná množina bola serializovaná do súboru \texttt{selected\_features.json}, ktorý je automaticky načítaný počas inferencie v real-time systéme.

\begin{table}[H]
\centering
\caption{Výsledky výberu príznakov pomocou Random Forest dôležitosti (RFI) a permutačnej dôležitosti (PI) pre syntetické dáta.}
\begin{tabular}{|l|c|c|}
\hline
Metrika & RFI Synthetic & PI Synthetic \\
\hline
smffunction\_sm\_n4sessionreportsucc & 0.0385 & 0.0 \\
pcffunction\_pa\_sessionnbr & 0.0347 & 0.0213 \\
pcffunction\_pa\_policysmassosucc & 0.0367 & 0.070 \\
smffunction\_sm\_pdusessioncreationreq & 0.0385 & 0.043 \\
smffunction\_sm\_qos\_flow\_nbr & 0.0398 & 0.047 \\
\hline
\end{tabular}
\label{table:feature-selection-synthetic}
\end{table}

\begin{table}[H]
\centering
\caption{Výsledky výberu príznakov pomocou Random Forest dôležitosti (RFI) a permutačnej dôležitosti (PI) pre reálne dáta.}
\begin{tabular}{|l|c|c|}
\hline
Metrika & RFI Real & PI Real \\
\hline
smffunction\_sm\_n4sessionreportsucc & 0.049 & 0.037 \\
pcffunction\_pa\_sessionnbr & 0.048 & 0.034 \\
pcffunction\_pa\_policysmassosucc & 0.049 & 0.016 \\
smffunction\_sm\_pdusessioncreationreq & 0.045 & 0.008 \\
smffunction\_sm\_qos\_flow\_nbr & 0.052 & 0.006 \\
\hline
\end{tabular}
\label{table:feature-selection-real}
\end{table}

Finálny výber príznakov bol následne overený pomocou troch nezávislých selekčných metód — Recursive Feature Elimination (RFE), Recursive Feature Elimination with Cross-Validation (RFECV) a Sequential Feature Selection (SFS). Hoci tieto metódy neboli použité ako primárne kritérium pri výbere metrických príznakov, slúžili ako validačný nástroj na potvrdenie robustnosti zvoleného výberu. Každá metrika zaradená do finálneho zoznamu bola zároveň označená ako relevantná aspoň jednou z týchto metód. Tento výsledok podporuje stabilitu a generalizačnú schopnosť modelu pri práci s dátami z rôznych domén a potvrdzuje, že výber nebol ovplyvnený len jednou technikou alebo šumom v trénovacích dátach.


\begin{table}[H]
\caption{Tabuľka ukazujúca príklad riadku zo syntetického datasetu.}
\begin{tabular}{ |p{2cm}|p{1.7cm}|p{1.6cm}|p{0.3cm}|p{2cm}|p{2cm}|p{2cm}|  }
\hline
 timestamp & amf sessions value & bearers active value & ... & application & lot type & current uc \\
 \hline
 2025-04-11 14:41:57  & 4.0  & 4.0 & ... & amf & registration & uc1  \\
 \hline
\end{tabular}
\label{table:dataset}
\end{table}


Shellové skripty boli prekonvertované do Unix formátu (LF), aby bola zaručená ich spustiteľnosť v Linuxových kontajneroch. Pri verziovaní bolo použité \texttt{core.autocrlf = input}, aby nedochádzalo k problémom s formátovaním riadkov na rôznych operačných systémoch.