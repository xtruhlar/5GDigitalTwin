\section{Formulácia problému}
V oblasti 5G sietí predstavujú DT perspektívny nástroj na bezpečné testovanie, monitorovanie a optimalizáciu správania siete. Vysoké nároky 5G na nízku latenciu, spoľahlivosť a flexibilitu riadenia zároveň vyžadujú schopnosť rýchlo identifikovať rôzne typy bežného správania, ale aj anomálie a útoky. Aby bolo možné efektívne trénovať klasifikačné modely bez rizika pre produkčnú infraštruktúru, je potrebné simulovať rôzne scenáre v kontrolovanom prostredí - DT.

Kľúčovým problémom však zostáva otázka, do akej miery môžu modely trénované výhradne na syntetických dátach zo simulovaných prostredí, ako Open5GS a UERANSIM, generalizovať na reálne siete, keďže výrazný nesúlad medzi syntetickými a reálnymi dátami môže obmedziť praktickú využiteľnosť takto vytvorených modelov.

Cieľom tejto bakalárskej práce je preto navrhnúť a implementovať jednoduché DT 5G siete a experimentálne overiť limity využitia syntetických Open5GS metrík na trénovanie klasifikačných modelov schopných rozpoznať reálne správanie siete.