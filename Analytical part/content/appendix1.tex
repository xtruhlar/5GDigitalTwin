\setcounter{figure}{0}
\setcounter{listing}{0}

\chapter{Používateľská príruška}
\label{appendix:userguide}
\pagenumbering{arabic}
\renewcommand*{\thepage}{A-\arabic{page}}

\begin{refsegment}

\section{Inštalácia}
Jednotlivé komponenty stiahneme z GitHub repozitára a následne ich zostavíme pomocou príkazov:
\begin{lstlisting}[language=bash,caption={Zostavovanie jednotlivých komponentov}, style=bashstyle]
# open5gs EPC/5GC komponenty
git clone https://github.com/herlesupreeth/docker_open5gs
cd docker_open5gs/base
docker build --no-cache --force-rm -t docker_open5gs .

# kamailio IMS komponenty
cd ../ims_base
docker build --no-cache --force-rm -t docker_kamailio .

# srsRAN_4G eNB + srsUE (4G+5G)
cd ../srslte
docker build --no-cache --force-rm -t docker_srslte .

# srsRAN_Project gNB
cd ../srsran
docker build --no-cache --force-rm -t docker_srsran .

# UERANSIM (gNB + UE)
cd ../ueransim
docker build --no-cache --force-rm -t docker_ueransim .
\end{lstlisting}

\subsection{NetData}
Nato aby sme mohli službu NetData integrovať do našeho projektu, je potrebné pridať túto časť kódu do \textbf{docker-compose} súboru:

\begin{lstlisting}[language=bash,caption={Integrácia služby NetData}, style=docker]
# --- Existujuci kod ---
  netdata:
    image: netdata/netdata:edge
    container_name: netdata
    ports:
      - "19999:19999"
    restart: always
    cap_add:
      - SYS_PTRACE
    security_opt:
      - apparmor:unconfined
    volumes:
      - netdataconfig:/etc/netdata
      - netdatalib:/var/lib/netdata
      - netdatacache:/var/cache/netdata
      - /etc/passwd:/host/etc/passwd:ro
      - /etc/group:/host/etc/group:ro
      - /etc/localtime:/etc/localtime:ro
      - /proc:/host/proc:ro
      - /sys:/host/sys:ro
      - /etc/os-release:/host/etc/os-release:ro
      - /var/log:/host/var/log:ro
      - /var/run/docker.sock:/var/run/docker.sock:ro
      - /run/dbus:/run/dbus:ro
    environment:
      - NETDATA_CLAIM_TOKEN=${NETDATA_CLAIM_TOKEN}
      - NETDATA_CLAIM_URL=${NETDATA_CLAIM_URL}
      - NETDATA_CLAIM_ROOMS=${NETDATA_CLAIM_ROOMS}
    networks:
      default:
        ipv4_address: ${NETDATA_IP}
# --- Existujuci kod ---
\end{lstlisting}

\section{Premenné prostredia}
V súbore \textbf{.env} sú definované globálne premenné, ktoré sú používane naprieč všetkými komponentami. Ďalej v súbore \textbf{.env} sú taktiež definované jednotlivé parametre UE zariadení, s ktorými v práci pracujeme.

\begin{lstlisting}[language=bash,caption={Súbor .env}, style=env]
MCC=001
MNC=01
TEST_NETWORK=172.22.0.0/24
DOCKER_HOST_IP=192.168.1.223

# MONGODB
MONGO_IP=172.22.0.2

# ...

# UERANSIM
NR_GNB_IP=172.22.0.23
NR_UE1_IP=172.22.0.24
UE1_IMSI=001011234567895
UE1_KI=8baf473f2f8fd09487cccbd7097c6862
UE1_OP=11111111111111111111111111111111

# ...

# NETDATA
NETDATA_IP=172.22.0.41
NETDATA_PORT=19999
NETDATA_CLAIM_TOKEN=[Token]
NETDATA_CLAIM_URL=https://app.netdata.cloud
NETDATA_CLAIM_ROOMS=[Room]
\end{lstlisting}

\section{Spustenie Open5gs Core komponentov}
Pre spustenie jednotlivých komponentov je vytvorený súbor vo formáte yaml, ktorý zabezpečuje správne spustenie všetkých častí v správnom poradí a so správnym nastavením. Aby sa premenné prostredia dostali na správne miesto v konfiguračncýb súboroch je potrebné zabezpečiť, aby o nich prostredie vedelo. To zabezpečíme pomocou nasledovných príkazov:

\begin{lstlisting}[language=bash,caption={Nastavenie zdroja premenných}, style=docker]
set -a
source .env
set +a
\end{lstlisting}

Tieto príkazy musíme použiť pri každej zmene súboru .env. Následne môžeme pomocou nasledovného príkazu spustiť Open5gs komponenty ako AMF, AUSF, BSF, SMF, UPF a ďalšie. Navyše okrem hlavných komponentov yaml súbor obsahuje aj služby s metrikami ako Prometheus, Grafana a NetData ako aj WebUI:

\begin{lstlisting}[language=bash,caption={Spustenie}, style=docker]
docker compose -f deploy-all.yaml up --build
\end{lstlisting}

\section{Vytváranie a spúšťanie zariadení (UE)}
Pre pridanie nového zariadenia (UE), je potrebné do súboru .env definovať následovné premenné:

\begin{lstlisting}[language=bash,caption={Súbor .env - Pridanie UE}, style=env]
UE${cislo}_IMEI=
UE${cislo}_IMEISV=
UE${cislo}_IMSI= 
UE${cislo}_KI=
UE${cislo}_OP= 
UE${cislo}_AMF=
\end{lstlisting}

Definované premenné je potrebné zaregistrovať pomocou Open5gs Webui. Bežne Webui beží na porte 9999. Do webového prehliadača zadáme adresu localhost:9999. Pre prístup do databázy UE je potrebné sa prihlásiť používateľským menom a heslom a následne registrovať vytvorené UE.

Meno: admin \\
Heslo: 1423

\section{Spustenie nr-gnb a nr-ue}
Pre spustenie komponentov UERANSIM - gNB a UE zariadení je potrebné ako prvé spustiť kontajner pre nr-gnb. Na to použijeme následovný príkaz:

\begin{lstlisting}[language=bash,caption={Spustenie nr-gnb}, style=docker]
docker compose -f nr-gnb.yaml up -d
\end{lstlisting}

Následne môžeme spustiť docker kontajner so zariadením, je potrebné zabezpečiť aby yaml súbor zariadenia bol v priečinku nr-UEs.
\begin{lstlisting}[language=bash,caption={Spustenie nr-ue}, style=docker]
docker compose -f nr-UEs/nr-ue1.yaml up --build -d
\end{lstlisting}

% Bibliography
\printbibliography[heading=referencessec,segment=\therefsegment,resetnumbers=true]

\end{refsegment}