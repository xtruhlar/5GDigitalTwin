\section{Tímová práca, diverzita a inklúzia}

Na vypracovaní tejto bakalárskej práce som síce pracoval samostatne, no viaceré časti riešenia by neboli realizovateľné bez odbornej podpory a spolupráce s ďalšími osobami a komunitami.

Počas celého vývoja som úzko spolupracoval so školiteľom Ing. Matejom Petríkom, ktorý mi poskytol dôležité usmernenia v oblasti DT, návrhu experimentov, spracovania dát a validácie výsledkov. V technických otázkach týkajúcich sa 5G sietí, konfigurácie Open5GS a správneho zberu metrických údajov zo siete som mal možnosť konzultovať s Ing. Matejom Janebom, ktorý sa aktívne podieľal na overovaní konfigurácie experimentálnej infraštruktúry a asistoval aj pri vytváraní reálneho datasetu v kontrolovanom laboratórnom prostredí.

Dôležitou súčasťou riešenia bola aj interakcia s komunitou vývojárov open-source nástroja UERANSIM. V rámci otvorenej komunikácie som kontaktoval hlavného autora projektu Aliho Güngöra so žiadosťou o objasnenie spracovania RRC Release signálov, ktoré v oficiálnej verzii nástroja absentovali. Tento typ spolupráce prispel k lepšiemu pochopeniu limitov dostupných nástrojov a ukázal význam komunitne vyvíjaného softvéru pri budovaní výskumných riešení.

Práca zohľadňuje princípy inklúzie tým, že všetky skripty a výstupy sú dokumentované, konfigurovateľné a prístupné pre širšiu výskumnú komunitu bez obmedzenia technologickej platformy či licencie. Prostredníctvom modularity riešenia, dôrazu na open-source a dokumentovanej kontajnerizácie je výstup tejto práce plne replikovateľný a využiteľný aj pre iných výskumníkov s rôznym zázemím a úrovňou odbornosti.
