\section{Tímová práca, diverzita a inklúzia}
%\par{
%Ak sa na projekte podieľajú odborníci z viacerých odborov, študent by mal implementovať procesy a opísať techniky pre zdieľanie úloh a (pochopenie) znalostí medzi rôznymi stranami. Mali by sa opísať problémy a stratégie na ich zmiernenie, aby sa zabezpečilo dokončenie projektu v stanovenom časovom harmonograme. Táto časť by mala obsahovať aj úvahy o diverzite a inklúzii. | ½-1 strana
%}
\par{
Teamwork and Knowledge Sharing: \\
- spolupráca s Matejom a Matejom

Diversity and Inclusion:
- Ageism - pomoc pre všetky vekové skupiny?
- Disabled people? hearing, vision ...
- môže im to DT nejko pomôcť?
- Prehľadná dokumentáia a návod pre možnú spoloprácu...
}

Spolupráca s konzultantom a inými odborníkmi: Môžete zdôrazniť, ako ste spolupracovali s konzultantom, odborníkmi alebo kolegami, aby ste získali spätnú väzbu alebo konzultovali technické aspekty projektu.

Diverzia myšlienok a prístupov: Aj keď ste projekt riešili individuálne, môžete opísať, ako ste zohľadnili rôzne perspektívy pri výbere riešení a metodík. Napríklad môžete zdôrazniť, že ste čerpali z rôznych zdrojov, ako sú články, výskumy a prípadové štúdie, aby ste vytvorili univerzálne použiteľný a inkluzívny návrh.

Prístupnosť a inklúzia výsledkov projektu: Opíšte, ako váš projekt môže byť prínosný pre širšiu komunitu používateľov. Napríklad, ak je váš digitálny dvojča schopné zlepšiť efektivitu nasadzovania 5G sietí, môže byť relevantný aj pre menej rozvinuté regióny alebo oblasti so zníženými zdrojmi.

Schopnosť prispôsobiť projekt tímovému pracovnému prostrediu: Môžete uviesť, že ste navrhli systém s ohľadom na modularitu a jednoduchú integráciu, čo umožňuje budúcu spoluprácu tímov na rozšírení projektu
