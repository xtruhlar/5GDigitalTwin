\chapter{Úvod}
\par{
%Študenti by mali načrtnúť problematiku, zdôrazniť súčasné výzvy a hlavné zásady a diskutovať o interdisciplinárnych aspektoch. ½ strany
Problematika digitálneho dvojčaťa (DT) je v súčasnosti jednou z najviac sa rozvíjajúcich oblastí IT \cite{DimensionOfDTAplication}, nachádzajúca svoje uplatnenie v priemysle \cite{manufacturing}, zdravotníctve \cite{siemens_helthcare}, ako aj v telekomunikáciách \cite{AplicationsOfDT}. Tieto virtuálne repliky fyzických objektov alebo systémov umožňujú simuláciu a predikciu správania, čo vedie k efektívnejším procesom a lepšiemu rozhodovaniu.
}

\par{
Hlavnými výzvami v oblasti DT pre 5G siete je zaistenie presnosti modelovania a predikcie správania siete v reálnom čase \cite{challenges_human_factor}. Zároveň ide o multidisciplinárnu výzvu, vyžadujúcu kombináciu poznatkov z oblasti telekomunikácií, strojového učenia a softvérového inžinierstva. Interdisciplinárny charakter témy zdôrazňuje potrebu prepojenia teoretických vedomostí s praktickými schopnosťami s cieľom prispieť k inovatívnym riešeniam v oblasti 5G sietí.
}
\par{
Práca si kladie za cieľ využiť dostupné nástroje, ako sú Open5GS \cite{open5gs}, UERANSIM \cite{ueransim} a srsRAN \cite{srsran}, na vytvorenie DT, ktoré dokáže predpovedať stav siete na základe aktuálnych a historických údajov. Tato schopnosť predikovať budúce stavy bude mať veľký prínos pri optimalizovaní sieťových zdrojov.
}