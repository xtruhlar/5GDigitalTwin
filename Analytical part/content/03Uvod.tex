\chapter{Úvod}
%Študenti by mali načrtnúť problematiku, zdôrazniť súčasné výzvy a hlavné zásady a diskutovať o interdisciplinárnych aspektoch. ½ strany
Technológia digitálneho dvojčaťa (Digital Twin, DT) predstavuje jeden z najperspektívnejších prístupov v oblasti moderného softvérového inžinierstva \cite{DimensionOfDTAplication} a telekomunikácií \cite{AplicationsOfDT}. DT umožňujú vytváranie verných virtuálnych modelov fyzických systémov, vďaka čomu je možné efektívnejšie sledovať, analyzovať a optimalizovať ich správanie v reálnom čase \cite{real_time}. Tieto technológie si už našli uplatnenie v oblasti priemyslu \cite{manufacturing}, zdravotníctva \cite{siemens_helthcare}, dopravy či energetiky, kde prispievajú k zvýšeniu efektivity či prediktívneho riadenia.

V kontexte mobilných sietí, najmä 5G, ponúkajú DT možnosť simulovať komplexné sieťové správanie, predvídať anomálie a optimalizovať správu sieťových zdrojov \cite{5gandbeyond}. 5G siete sú charakteristické vysokou dynamikou, rozmanitosťou zariadení a rôznorodosťou požiadaviek na kvalitu služieb (QoS), čo kladie vysoké nároky na monitorovanie a riadenie \cite{AplicationsOfDT}. DT môže v tomto prostredí slúžiť ako nástroj na testovanie rôznych scenárov bez zásahu do reálnej prevádzky siete a zároveň znížiť bezpečnostné riziká, ako sú potenciálne výpadky služieb alebo strata integrity dát pri priamom testovaní na živej sieti.

Hlavnou výzvou pri tvorbe DT pre 5G siete je zabezpečiť, aby simulované správanie čo najvernejšie odrážalo reálne procesy v sieti. Problémom je najmä otázka, či modely trénované výlučne na syntetických dátach dokážu spoľahlivo klasifikovať správanie v reálnych podmienkach. Táto bakalárska práca si kladie za cieľ preskúmať túto problematiku vytvorením a implementáciou jednoduchého DT 5G siete využívajúceho nástroje Open5GS \cite{open5gs} a UERANSIM \cite{ueransim}.