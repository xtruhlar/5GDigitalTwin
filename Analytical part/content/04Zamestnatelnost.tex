\section{Zamestnateľnosť}
Táto bakalárska práca, zameraná na rozvoj teoretických znalostí v oblasti digitálnych dvojčiat v spojení s praktickou implementáciou a optimalizáciou 5G sietí, vedie k rozšíreniu zručností vo viacerých kľúčových oblastiach technologického sektora. V neposlednom rade strojové učenie, použité na predikciu správania implementovaného digitálneho dvojčaťa, zasahuje aj do oblasti dátovej vedy.

Vďaka formátu práce autori prejdú celým cyklom realizácie projektu, od prieskumu technológií cez návrh až po implementáciu a testovanie. Týmto získajú ucelený a komplexný pohľad na vývoj a riadenie softvérových projektov, ako aj na plánovanie, organizáciu a efektívnu komunikáciu.

Takáto kombinácia technických, projektových a komunikačných schopností môže významne zvýšiť hodnotu autorov na trhu práce, najmä v budúcnosti, keďže problematika digitálnych dvojčiat a 5G sietí je stále viac žiadaná a nachádza uplatnenie v rôznych odvetviach.