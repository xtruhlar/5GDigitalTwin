\chapter{Záver}

V rámci tejto bakalárskej práce bol navrhnutý a implementovaný základný model digitálneho dvojčaťa (DT) pre 5G sieť, ktorý umožňuje simuláciu správania siete a zber dát z fyzickej i simulovanej infraštruktúry. Použitím nástrojov Open5GS a UERANSIM bol vytvorený kontrolovaný experimentálny priestor, v ktorom bolo možné testovať rôzne používateľské scenáre bez zásahu do reálnej siete. Výsledky ukazujú, že klasifikátory trénované výhradne na syntetických dátach dosahovali výrazne horšiu presnosť pri aplikácii na reálne dáta (~16\%), v porovnaní so ~93\% presnosťou na dátach zo simulácie. Tento pokles jasne signalizuje rozdiel medzi reálnym a simulovaným sieťovým správaním a zdôrazňuje potrebu spoľahlivých metód prenosu znalostí medzi týmito dvoma doménami.

Tieto zistenia poukazujú na viaceré limity implementovaného prístupu, ktoré je potrebné otvorene pomenovať. V prvom rade sa ukázalo, že syntetické metriky generované pomocou Open5GS majú obmedzenú schopnosť zachytiť komplexitu reálneho sieťového správania. Chýba im detailná reprezentácia QoS mechanizmov, realistický prenos na aplikačnej vrstve, ako aj podpora mobility, čo výrazne znižuje použiteľnosť týchto simulácií ako základ pre trénovanie modelov s dobrou generalizačnou schopnosťou. Hoci boli testované viaceré architektúry klasifikačných modelov (vrátane attention mechanizmu a batch normalizácie), výkonnosť na reálnych dátach ostávala nízka, čo naznačuje, že problém spočíva predovšetkým v obmedzenej kvalite dát. Navyše, použité metriky sa týkali výlučne core vrstvy, pričom chýbali detailnejšie údaje z RAN vrstvy, ako napríklad parametre rádiového signálu alebo využitie fyzických kanálov. Z pohľadu reálneho experimentálneho zberu bola práca obmedzená na malý počet zariadení, krátke trvanie meraní, čo mohlo ovplyvniť robustnosť výstupov. Hoci väčšina simulácií reprezentovala štandardné prípady správania, v jednom z prípadov bol zaradený aj anomálny scenár s odmietnutím pripojenia na základe neplatného IMSI, čo pomohlo zvýšiť rozmanitosť dát.

Práca ukazuje, že real-time klasifikácia a online trénovanie modelov je technicky realizovateľné aj v rámci jednoduchého DT, hoci zatiaľ len pri použití syntetických dát. Výsledky tejto práce zároveň naznačujú, že DT môžu nájsť uplatnenie v oblasti riadenia a optimalizácie 5G sietí – predovšetkým ako nástroj na bezpečné testovanie sieťových zmien, včasné odhaľovanie anomálií a trénovanie systémov umelej inteligencie mimo produkčného prostredia. Do budúcnosti by mal byť výskum orientovaný na zvýšenie vernosti simulácií a zlepšenie kvality dát použitých na trénovanie modelov. Za účelom premostenia medzi syntetickými a reálnymi dátami sa ako perspektívna javí kombinácia metrických údajov z Open5GS s dátami zachytenými v podobe paketov (PCAP), čo umožní extrahovať dotatočné charakteristiky sieťového správania a vrstviť informácie naprieč L2–L7. Zároveň je potrebné preskúmať možnosti aplikácie pokročilých doménovo adaptačných techník, ktoré by mohli znížiť závislosť na rozsiahlych reálnych datasetoch.

Výzvou do budúcnosti ostáva rozšíriť real-time prístup aj na reálne sieťové prostredie a uzavrieť spätnú regulačnú slučku – teda umožniť systému nielen detegovať anomálie v reálnom čase, ale aj autonómne reagovať vhodnou rekonfiguráciou siete alebo notifikáciou správcu. Takýto obojsmerný tok dát medzi fyzickým a digitálnym prostredím predstavuje ďalší krok smerom k autonómnym, adaptívnym sieťam piatej generácie.