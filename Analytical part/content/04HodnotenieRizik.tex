\section{Hodnotenie rizík}
Implementácia digitálneho dvojčaťa 5G siete spolu s modelom ML prináša rôzne výzvy, ktoré môžu ovplyvniť presnosť klasifikácie, stabilitu a efektívnosť celého systému. 

Nedostatočná kvalita vstupných dát je jedným z najpravdepodobnejších rizík, ktoré môžu viesť k nesprávnym výsledkom modelu. Chyby v dátach alebo ich nereprezentatívnosť, napríklad pri modeloch sieťovej prevádzky (traffic patterns), môžu narušiť presnosť predpovedí \cite{ML_traffic}. Formou zmiernenia je testovanie na rôznych dátových scenároch a aplikácia metód ako krížová validácia a ladenie hyperparametrov, ktoré minimalizujú riziko chýb spojených s podtrénovaním (underfitting) a pretrénovaním (overfitting) \cite{Nguyen}.

Zber kvalitných dát môže byť problematický, pretože mnohé scenáre je potrebné zachytiť v laboratóriu. Bez dostatočných dát môže model generovať neadekvátne predpovede, ktoré nebudé reprezentovať skutočnosť. Riešením môže byť generovanie syntetických údajov \cite{data_generating} a využitie dostupných datasetov z iných projektov \cite{datasets_telecom}, ktoré môžu čiastočne nahradiť reálne dáta a napomôcť k presnejším predpovediam.

Nezvyčajné scenáre, ako vysoké zaťaženie siete (peak loads), neštandardné správanie používateľov či poveternostné podmienky môžu narušiť schopnosť modelu adaptovať sa \cite{challenges_human_factor}. Ak tieto scenáre neobsahujú trénovacie dáta, model nemusí byť pripravený na takéto situácie. Vzhľadom na čas, ktorý máme na získanie a predspracovanie dát, tento problém nemusí byť vo finálnej implementácií vyriešený dostatočne, a preto jeho vyriešenie vyžaduje ďalšiu prácu a zber dát.

Kompatibilita systémov ako srsRAN, Open5GS, UERANSIM a nástrojov Docker nie je vždy automatizovaná, čo môže ovplyvniť celkovú integráciu práce \cite{challenges_human_factor}. Z tohoto dôvodu je vhodné začínať implementáciu na malých izolovaných komponentoch, za neustáleho testovania. Takéto testovanie, a celkový vývoj DT sú časovo veľmi náročné, čo potvrdili aj autori v \cite{USAirForce}. Tento fakt môže negatívne vplývať na výsledok celej práce, nakoľko čas je jedným z kľúčových faktorov, ktoré majú vplyv na množstvo a kvalitu pozberaných dát určených na trénovanie ML modelu.

Konfigurácia siete môže odhaliť citlivé informácie o 5G infraštruktúre či porušenie GDPR \cite{big-data-problems} pri nechcenom zachytení údajov o používateľoch \cite{challenges-technol}. Únik takýchto dát by ohrozil nielen bezpečnosť projektu, ale aj reálnej siete \cite{Dt_Iot_data_worry_about}. Používanie .env súborov na uchovávanie citlivých premenných a simulovanie siete s fiktívnymi údajmi výrazne znižuje riziko úniku.