\section{Udržateľnosť a environmentálny dopad}
%Študenti by mali opísať, ako by implementovali opatrenia na zabezpečenie udržateľnosti svojho produktu alebo služby počas životného cyklu. | 1-2 strany
Implementácia opatrení na zabezpečenie udržateľnosti projektu a minimalizácia jeho environmentálneho dopadu sú kľúčové pre zaistenie dlhodobej hodnoty, spoločenského prínosu a efektivity vypracovania tejto práce. Práca je navrhnutá s dôrazom na efektívne využívanie zdrojov, udržateľný softvérový a hardvérový návrh a životný cyklus 5G siete, ktorý minimalizuje potrebu používania fyzických zdrojov.

Jednou z hlavných prínosov DT je možnosť predikcie a optimalizácie, pričom ak je DT zostrojené správne, môže dopomôcť k redukcií spotreby elektrickej energie. \cite{DT_edge_networks_IoT}. Predikcia budúceho stavu siete umožňuje taktiež lepšiu správu záťaže (traffic load) a preťaženia (congestion), čo napomáha k znižovaniu nadmernej spotreby energie \cite{malaysia_enviro}. Navyše, DT eliminuje potrebu testovania na fyzických zariadeniach, čím sa minimalizuje spotreba rôznych materiálov \cite{enviro_raw_materials} a času potrebného na fyzické experimenty. Tento prístup je obzvlášť užitočný v prípade nasadzovania a testovania 5G technológií v oblastiach s nerozvinutou infraštruktúrou a obmedzenými výrobnými zdrojmi \cite{huaweii_i_cities}.

Vývoj softvéru bol orientovaný na maximálnu efektivitu, čo zahŕňa optimalizáciu kódu na zníženie spotreby energie počas behu aplikácie a nasadenie projektu v prostredí Docker, čo umožňuje rýchlejšiu konfiguráciu a škálovateľnosť zariadení. Tieto opatrenia nielen znižujú environmentálny dopad \cite{docker_enviro}\cite{docker_enviro_2}, ale aj zvyšujú celkovú udržateľnosť projektu.

Modulárny dizajn \cite{modular_sw} projektu zabezpečuje, že aktualizácie a údržba nemajú vplyv na celkovú funkčnosť systému. Tento prístup znižuje potrebu kompletného prekonfigurovania alebo fyzických zásahov do chodu programu, čo prispieva k dlhodobej udržateľnosti. Takýto dizajn môže mať pozitívny vplyv na životné prostredie \cite{modular_sw} (Green Design).

Ako je vyššie uvedené, použitie prediktívnych modelov v DT môže viesť k zásadným environmentálnym prínosom \cite{enviro}. Okrem zníženia zaťaženie fyzickej infraštruktúry a menej častých aktualizácie fyzického hardvéru, môže mať za následok aj nižšiu spotrebu zdrojov a menšiu produkciu odpadu \cite{enviro_raw_materials}. Predikčné modely teda umožňujú efektívnejšie rozhodovanie s pozitívnym dopadom na životné prostredie.