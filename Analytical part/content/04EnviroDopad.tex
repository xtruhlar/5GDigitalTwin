\section{Udržateľnosť a environmentálny dopad}
DT predstavujú konceptuálne aj prakticky významný prístup k optimalizácii prevádzkových procesov, ktorý v porovnaní s fyzickým testovaním výrazne znižuje spotrebu zdrojov, produkciu odpadu a prevádzkové riziká \cite{enviro_raw_materials}. V kontexte návrhu DT 5G siete, implementovaného v rámci tejto práce, bola environmentálna udržateľnosť reflektovaná v niekoľkých rovinách.

Z pohľadu výpočtovej náročnosti bola celá infraštruktúra navrhnutá s cieľom minimalizovať energetickú záťaž. Všetky komponenty boli nasadené na lokálnom výpočtovom uzle bez použitia centralizovaných cloudových služieb alebo akcelerátorov typu GPU. Kombinácia kontajnerizovaných služieb (Docker Compose) a open-source softvérových nástrojov (Open5GS, UERANSIM, Prometheus, Grafana) umožnila zabezpečiť nízku spotrebu zdrojov pri zachovaní dostatočnej funkcionality pre experimentálne overenie konceptu \cite{docker_enviro, docker_enviro_2}.

V širšom zmysle sú DT považované za technológiu s potenciálom znižovať uhlíkovú stopu prostredníctvom presnejšej správy sietí, prediktívnej údržby a minimalizácie potreby fyzických zásahov do infraštruktúry \cite{DT_edge_networks_IoT, enviro}. Aj keď prediktívne modelovanie nebolo predmetom tejto práce, implementovaný systém poskytuje základ, na ktorom je možné budovať systémy podporujúce rozhodovanie s pozitívnym environmentálnym dopadom \cite{malaysia_enviro}.

Navyše, architektúra riešenia je modularizovaná, čo umožňuje selektívnu údržbu a výmenu komponentov bez nutnosti reinštalácie celého systému. Takýto návrhový prístup je v súlade s princípmi zeleného softvérového inžinierstva, ktoré zdôrazňujú znižovanie energetických nárokov a podporu dlhodobej životnosti softvérových riešení \cite{modular_sw}.

Navrhnutý systém teda nielenže zohľadňuje technologickú efektivitu a experimentálnu hodnotu, ale zároveň reflektuje požiadavky na environmentálnu udržateľnosť v oblasti návrhu a testovania moderných komunikačných systémov.